% Created 2016-04-17 Sun 15:35
\documentclass[11pt]{article}
\usepackage[utf8]{inputenc}
\usepackage[T1]{fontenc}
\usepackage{fixltx2e}
\usepackage{graphicx}
\usepackage{grffile}
\usepackage{longtable}
\usepackage{wrapfig}
\usepackage{rotating}
\usepackage[normalem]{ulem}
\usepackage{amsmath}
\usepackage{textcomp}
\usepackage{amssymb}
\usepackage{capt-of}
\usepackage{hyperref}
\author{Mike Zamansky}
\date{\today}
\title{}
\hypersetup{
 pdfauthor={Mike Zamansky},
 pdftitle={},
 pdfkeywords={},
 pdfsubject={},
 pdfcreator={Emacs 25.1.50.1 (Org mode 8.3.4)}, 
 pdflang={English}}
\begin{document}

\tableofcontents


\section{}
\label{sec:orgheadline1}
Switched over from Jekyll to Nikola yesterday.

One of the primary reasons was that I got fed up with managing a Ruby
development environment across all my machines. In spite of my
rantings, I'm sure Ruby adn Ruby installas are fine - it's just
something that I'd have to deal with on a recurring basis for one
specific task - other than for Jekyll blogging, I don't use Ruby.

So while a platform based on Python - generally my go to language,
made sense, the other driving force for the switch was the fact that
\href{https://www.gnu.org/software/emacs/}{Emacs} and \href{http://orgmode.org/}{org-mode} have become such major parts of my workflow and
productivity.

What do I use org-mode for?
\begin{itemize}
\item lesson planning
\item document preparation (instead of \LaTeX  )
\end{itemize}

\frac{2}{3}
\LaTeX
world
\end{document}